\nomenclature{$\mathcal{L}$}{Luminosity at a collision point}
\nomenclature{$\sigma_{p}$}{Production cross section for a particle reaction}
\nomenclature{$N_{b}$}{Bunch population. b replaced by 1 or 2 for the case of colliding beams}
\nomenclature{$f_{rev}$}{Revolution frequency of the accelerator}
\nomenclature{$n_{b}$}{The number of bunches in the accelerator. In the case of colliding beams it it the number of colliding bunches}
\nomenclature{$\sigma_{x,y}$}{Gaussian bunch sigma for the transverse cross-section in the horizontal and vertical planes}
\nomenclature{$\sigma_{x,y}$}{Gaussian bunch sigma for the transverse cross-section in the horizontal and vertical planes}
\nomenclature{$e$}{Charge of an electron}
\nomenclature{$I_{b}$}{DC beam current}
\nomenclature{$(\rho , \theta , z)$}{Coordinate system of an ideal on-momentum particle}
\nomenclature{$(x , y , z)$}{Coordinate system comoving with the on-momentum particle}
\nomenclature{$q$}{Charge of a particle}
\nomenclature{$\mathbf{E} = (E_{x}, E_{y}, E_{z})$}{The electric field components relative to the coordinate system following the on-momentum particle}
\nomenclature{$\mathbf{B} = (B_{x}, B_{y}, B_{z})$}{The magnetic induction (referred to as the magnetic field in the text due to convention) components relative to the coordinate system following the on momentum particle}
\nomenclature{$\mathbf{v}$}{Particle velocity}
\nomenclature{$m$}{Particle Mass}
\nomenclature{$\gamma$}{Relativistic factor}
\nomenclature{$\beta$}{Particle velocity normalised by the speed of light}
\nomenclature{$p$}{Particle momentum}
\nomenclature{$k_{f}$}{Focal strength of a quadrupole magnet}
\nomenclature{$c$}{Speed of light in vacuum($=$299792458 m/s)}
\nomenclature{$\omega$}{Angular velocity or angular frequency}
\nomenclature{$k$}{Wave number}
\nomenclature{$(x_{1}, y_{1},z_{1})$}{The coordinates of the source or inducing particle of a wakefield}
\nomenclature{$(x_{2}, y_{2}, z_{2})$}{The coordinates of the test or witness particle of a wakefield}
\nomenclature{$q_{1}$}{Charge of the source or inducing particle of a wakefield}
\nomenclature{$q_{1}$}{Charge of the test or witness particle of a wakefield}
\nomenclature{$E_{\parallel}$}{The longitudinal component of an electric field}
\nomenclature{$\mathbf{E}_{\perp}$}{The perpendicular components of an electric field}
\nomenclature{$B_{\parallel}$}{The longitudinal component of a magnetic field}
\nomenclature{$\mathbf{B}_{\perp}$}{The perpendicular components of a magnetic field}
\nomenclature{$F_{\parallel}$}{The longitudinal component of a force}
\nomenclature{$\mathbf{F}_{\perp}$}{The perpendicular components of a force}
\nomenclature{$\tau}$}{The time between the source and witness particle in the comoving reference frame}
\nomenclature{$k_{loss}$}{The loss factor of a single particle wakefield}
\nomenclature{$w_{\parallel}$}{The longitudinal wake function}
\nomenclature{$W_{\parallel}$}{The longitudinal wake function of a bunch or wakepotential}
\nomenclature{$w_{\perp}$}{The transverse wake function}
\nomenclature{$W_{\perp}$}{The transverse wake function of a bunch or trasnverse wakepotential}
\nomenclature{$j$}{imaginary number $-\sqrt{-1}$}
\nomenclature{$Z_{\parallel}$}{The total longitudinal impedance}
\nomenclature{$Z_{\perp}$}{The total transverse impedance}
\nomenclature{$Z_{dip, x}$ or $Z_{dipolar, x}$}{The horizontal dipolar or driving impedance}
\nomenclature{$Z_{dip, y}$ or $Z_{dipolar, y}$}{The vertical dipolar or driving impedance}
\nomenclature{$Z_{quad, x}$ or $Z_{quadrupolar, x}$}{The horizontal quadrupolar or detuning impedance}
\nomenclature{$Z_{quad, y}$ or $Z_{quadrupolar, y}$}{The vertical quadrupolar or detuning impedance}
\nomenclature{$Z_{const, x}$ or $Z_{constant, x}$}{The horizontal constant impedance}
\nomenclature{$Z_{const, y}$ or $Z_{constant, y}$}{The vertical constant impedance}
\nomenclature{$Z_{0}$}{Impedance of free space}
\nomenclature{$\epsilon_{0}$}{The permitivitty of free space}
\nomenclature{$\mu_{0}$}{The permeability of free space}
\nomenclature{$R_{s}$}{The shunt impedance of a resonant impedance}
\nomenclature{$Q$}{The quality factor of a resonant impedance}
\nomenclature{$f_{res}$}{The resonant frequency of a resonant impedance}
\nomenclature{$\omega_{res}$}{The resonant angular frequency of a resonant impedance}
\nomenclature{$\epsilon_{r}$}{The relative permitivitty of a material}
\nomenclature{$\mu_{r}$}{The relative permeability of a material}
\nomenclature{$\mu^{'}$}{The real component the complex permeability of a material}
\nomenclature{$\mu^{"}$}{The imaginary component the complex permeability of a material}
\nomenclature{$\epsilon^{'}$}{The real component the complex permitivitty of a material}
\nomenclature{$\epsilon^{"}$}{The imaginary component the complex permitivitty of a material}
\nomenclature{$L$}{The inductance of an RLC circuit}
\nomenclature{$C$}{The capacitance of an RLC circuit}
\nomenclature{$P_{surf}$}{The power loss due to ohmic losses on the surface of a cavity}
\nomenclature{$P_{loss}$}{The power lost by a beam due to interaction with a beam impedance}
\nomenclature{$V_{acc}$}{The voltage experienced by a witness particle traversing a cavity}
\nomenclature{$W$}{The energy stored in a cavity due to a given cavity mode}
\nomenclature{$\delta$}{The skin depth of a material}
\nomenclature{$\sigma$}{The conductivity of a material}
\nomenclature{$\rho$}{The resistivity of a material}
\nomenclature{$Q_{0}$}{Unperturbed betatron tune}
\nomenclature{$Q_{pert}$}{Part of the betatron tune caused by a perturbing force}
\nomenclature{$t_{b}$}{The bunch length in time ($4\sigma$ length for Gaussian distributions)}
\nomenclature{$\sigma_{z}$}{The bunch length in distance ($4\sigma$ length for Gaussian distributions)}
\nomenclature{$\rho (t)$}{Time domain longitudinal distribution}
\nomenclature{$\lambda (f)$}{Frequency domain longitudinal distribution}
\nomenclature{$\lambda$}{Wavelength}
\nomenclature{$\Re{}e$}{The real component of a number or function}
\nomenclature{$\Im{}m$}{The imaginary component of a number or function}
\nomenclature{$S_{21}$}{The transmission parameter for an RF system}
\nomenclature{$S_{21,DUT}$}{The transmission parameter through a device under test}
\nomenclature{$T_{c}$}{Curie temperature of a ferrite}

