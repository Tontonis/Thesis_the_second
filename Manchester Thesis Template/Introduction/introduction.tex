The Standard Model of particle physics has successfully described the weak, electromagnetic and strong interactions by requiring symmetries under specific gauge transformations. The electroweak symmetry is spontaneously broken by the Higgs mechanism so that the particles acquire mass. This mechanism requires the existence of a massive scalar particle, the Higgs boson, but the mass of this new particle is not predicted by the model. The discovery of the Higgs boson remains one of the main priorities in particle physics. 

The physics potential of forward proton tagging at the Large Hadron Collider  (LHC) 
has received a great deal of attention in recent years. 
The LHC is a proton-proton collider with a centre-of-mass energy of 14 TeV.
In normal circumstances at the LHC, the protons break up during the interaction.  
In the central exclusive process however, the protons lose a small fraction of their momentum, remain intact, and all of the lost momentum goes into the production of the central system. 

The central exclusive process has a number of desirable properties. Firstly, there is no underlying event caused by proton dissociation and the environment is, for a hadron collider, remarkably clean. Furthermore, if the protons are detected and the momentum loss measured, it is possible to obtain an excellent resolution on the mass of the central system. If the central system is a resonance, such as the Higgs boson, then the mass can be determined accurately regardless of the decay channel. 

The work in this thesis focuses on evaluating the physics potential of the central exclusive process. In chapter two, a brief review of the Standard Model is followed by a description of the Durham model of central exclusive production. Chapter three introduces the LHC and the ATLAS experiment, which is a general purpose detector capable of observing a variety of physics signals at the LHC. The chapter concludes with a description of FP420, which is a system of forward detectors for the LHC.

The ExHuME event generator, which allows computer simulation of the central exclusive process, is described in chapter four. The magnitude of the di-photon cross section that could be observed using ATLAS is predicted. ExHuME is then used in chapter five to evaluate the Standard Model Higgs boson discovery potential, in the $b \bar{b}$ decay channel, using forward proton detectors at ATLAS. The analysis is also extended to the lightest Higgs boson in the intense coupling region of the Minimal Supersymmetric Standard Model. In chapter six, central exclusive di-jet production is discussed at the Fermilab Tevatron, which is a $p \bar{p}$ collider with a centre-of-mass energy of 1.96 TeV. Chapter seven summarises the results of the previous chapters.
