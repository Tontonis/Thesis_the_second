\section{Linear Transverse Dynamics}
\label{sec:linTransMotions}

Here we shall introduce the basics of linear transverse beam dynamics. This introduction follows the formalism introduced in \cite{Holzer:TransDyn}. If we consider a charged particle of mass $m$ and charge $e$ moving with velocity $v$ along the ideal orbit under the effect of the magnetic field $B$ we can see that the Lorentz force ($F_{L} = qvB$) and the centrifugal force ($F_{cen} = \frac{\gamma m v^{2}}{\rho}$) are equal, leading to the relation

\begin{equation}
\frac{p}{e} = B \rho
\label{eqn:beam-rigid}
\end{equation}

where $p$ is the particles momentum. $B \rho$ is referred to as the beam rigidity, denoting the constant term necessary to keep a particle on the ideal orbit. 

If we consider the motion of a particle transversally displaced from the ideal orbit particle with a position $(x, y)$ we must consider the variation of the magnetic field in the transverse directions. If we consider first just the horizontal motion, we can carry out a Taylor expansion of the field which gives us

\begin{equation}
B_{y} \left( x \right) = B_{y0} + \frac{dB_{y}}{dx} x + \frac{1}{2!}\frac{d^{2}B_{y}}{dx^{2}} x^{2} + \frac{1}{3!} \frac{d^{3}B_{y}}{dx^{3}} x^{3} + ...
\end{equation}

which is then normalised by $p/e$

\begin{equation}
\frac{B_{y} \left( x \right)}{p/e} = \frac{B_{y0}}{B_{y0}\rho} + \frac{1}{p/e}\frac{dB_{y}}{dx} x + \frac{1}{p/e}\frac{1}{2!}\frac{d^{2}B_{y}}{dx^{2}} x^{2} + \frac{1}{p/e}\frac{1}{3!} \frac{d^{3}B_{y}}{dx^{3}} x^{3} + ...
\end{equation}

For the time being we shall consider only small displacements in $x$ and $y$, thus shall consider only the linear terms, giving

\begin{align}
\frac{B_{y} \left( x \right)}{p/e} &= \frac{1}{\rho} + \frac{1}{p/e}\frac{dB_{y}}{dx} x \\
\frac{B_{y} \left( x \right)}{p/e} &= \frac{1}{\rho} + k_{f} x
\end{align}

where $k_{f} = \frac{g}{p/e}$, $g$ is the gradient of a quadrupole magnet (a magnet which has a magnetic field strength linearly proportional to displacement). This indicates that seperation of the function of magnets is a valid method of controlling the particles, using dipole and quadrupole fields. Next, to derive the equations of motion we shall consider the equation for radial acceleration $a_{r}$ for a particle undergoing rotational acceleration in the defined coordinate system

\begin{equation}
a_{r} = \frac{d^{2}\rho}{dt^{2}} - \rho \left( \frac{d\theta}{dt} \right)^{2}.
\end{equation}

For the particle on the ideal orbit $\rho = constant$, so $\frac{d \rho}{dt} = 0$, leading to the accelerative force being derived to be

\begin{equation}
F_{x} = m \rho \left( \frac{d\theta}{dt} \right)^{2} = m\rho \omega^{2} = \frac{mv^{2}}{\rho}
\end{equation} 

For a general particle, $\rho \rightarrow \rho + x$, leading to

\begin{equation}
F_{x} = m\frac{d^{2}}{dt^{2}} \left( x+ \rho \right) - \frac{mv^{2}}{x + \rho} = eB_{y} v.
\end{equation}

As $\rho = constant$, $\frac{d^{2}}{dt^{2}} ( x+ \rho )$ becomes $\frac{d^{2}x}{dt^{2}}$. In addition we are dealing with displacements of $x \ll \rho$, so we may approximate $\frac{1}{x+\rho} \approx \frac{1}{\rho}(1-\frac{x}{\rho})$ from the Taylor expansion and taking linear terms in $x$. Thus we acquire the equation

\begin{equation}
 m\frac{d^{2}x}{dt^{2}}  - \frac{mv^{2}}{\rho}\left( 1 - \frac{x}{\rho} \right) = eB_{y} v.
\end{equation}

Subsequently we subsitute the magnetic field $B_{y} = B_{y0} + \frac{dB_{y}}{dx} x$ and arrive at

\begin{equation}
 m\frac{d^{2}x}{dt^{2}}  - \frac{mv^{2}}{\rho}\left( 1 - \frac{x}{\rho} \right) = e v \left[ B_{y0} + \frac{dB_{y}}{dx} x \right].
\end{equation}

By dividing throught by the mass m

\begin{equation}
 \frac{d^{2}x}{dt^{2}}  - \frac{v^{2}}{\rho}\left( 1 - \frac{x}{\rho} \right) = \frac{e v B_{y0}}{m} + \frac{ev}{m}\frac{dB_{y}}{dx} x.
\end{equation}

Next we transform the coordinate from $t \rightarrow z$, such that 

\begin{equation}
\frac{d^{2}x}{dt^{2}} = \frac{d}{dt} \left( \frac{dx}{dz} \frac{dz}{dt} \right) = \frac{d}{dz} \left( \frac{dx}{dz} \frac{dz}{dt} \right) \frac{dz}{dt} = \frac{d^{2}x}{dz^{2}}\left( \frac{dz}{dt} \right)^{2} + \frac{dx}{dz} \frac{d}{dz} \left( \frac{dz}{dt} \right) \frac{dz}{dt}.
\end{equation}

It can be seen that $\frac{dz}{dt} = v$, which is kept constant in this treatment. Therefore

\begin{equation}
\frac{d^{2}x}{dt^{2}} = \frac{d^{2}x}{dz^{2}} v^{2}.
\end{equation}

Thus we arrive at the equation

\begin{equation}
\frac{d^{2}x}{dz^{2}} v^{2} - \frac{v^{2}}{\rho}\left( 1 - \frac{x}{\rho} \right) = \frac{e v B_{y0}}{m} + \frac{ev}{m}\frac{dB_{y}}{dx} x.
\end{equation}

We then subsequently normalise by momentum and rearrange slightly to acquire

\begin{equation}
\frac{d^{2}x}{dz^{2}} - \frac{1}{\rho} + \frac{x}{\rho^{2}}  = \frac{ B_{y0}}{p/e} + \frac{1}{p/e}\frac{dB_{y}}{dx} x.
\end{equation}

We then remember the definition of the beam rigidity (see Eqn.~\ref{eqn:beam-rigid}) and the definition of the quadrupolar field strength arriving at

\begin{align}
\frac{d^{2}x}{dz^{2}} - \frac{1}{\rho} + \frac{x}{\rho^{2}}  = - \frac{1}{\rho} + k_{f} x \nonumber \\ 
\frac{d^{2}x}{dz^{2}} + \frac{x}{\rho^{2}}  = k_{f} x \nonumber \\ 
\frac{d^{2}x}{dz^{2}} + x \left( \frac{1}{\rho^{2}} - k_{f}\right)  = 0. \label{eqn:horz-eqn-motion}
\end{align}

For the vertical plane the equation is the same, except that the focusing term is negative and there is no $\rho$ term, giving

\begin{equation}
\frac{d^{2}y}{dz^{2}} +  k_{f} y  = 0.
\label{eqn:vert-eqn-motion}
\end{equation}

It is interesting to note that there is a focusing term due to the dipolar magnet term. Historically this is the weak focusing effect, as opposed to the strong focusing regime of using quadrupole magnets. It can be seen that these equations are simply the equations of motion of a pseudo-harmonic oscillator with spring constants $K_{x/y}$ given by

\begin{align*}
K_{x} = \frac{1}{\rho^{2}} - k_{f} \\
K_{y} = k_{f}
\end{align*}

leading to a general solution 

\begin{equation}
x/y \left( s \right) = a_{1} cos \left( \sqrt{K_{x/y}} s \right) + a_{2} sin \left( \sqrt{K_{x/y}} s \right).
\end{equation}

For a modern particle accelerator using focusing and defocusing magnets arranged in a periodic lattice (i.e. so that $k(s)$ becomes a periodic function such that $k_{f}(s+L) = k_{f}(s)$ where L is the length of the period of the lattice), and applying initial conditions $x(0) = x_{0}, x^{'}(0) = x^{'}_{0}$ it is possible to finally arrive at the solution for the motion of a particle through a periodic lattice

\begin{equation}
x \left( s \right) = \sqrt{\epsilon} \sqrt{\beta \left( s \right) } cos \left( \psi \left( s \right) + \phi \right)
\end{equation}

where $\epsilon$ and $\phi$ are integration constants determined by the initial conditions, $\beta (s)$ is a periodic function determined by the focusing properties of the lattice. $ \psi (s)$ is the phase advance of the oscillation, again determined by the lattice design parameters. These parameters define beam parameters, $\epsilon$ being the beam emittance, the preserved quantity in transverse phase space, $\beta$ the betatron function, which is a property of the lattice and determines the physical size of the beam. One key important figure acquired from this derivation is the tune (also called the betatron tune) of the accelerator, that is the number of betatron oscillations per turn, given by

\begin{equation}
Q_{0} = \frac{1}{2\pi} \oint \frac{ds}{\beta \left( s \right)}.
\end{equation}

\subsubsection{Off-Momentum Particles}

During the operation of a particle accelerator there is often have a spread of particle energies due to dynamics in the longitudinal plane \cite{Leduff:LongDyn}. This gives rise to a momentum spread $\Delta p/p$ in the particle momentum, where $\Delta p = p - p_{0}$, $p_{0}$ being the on-path momentum. When considered in Eqn.~\ref{eqn:horz-eqn-motion} this gives rise to an inhomogenous differential equation of the form

\begin{equation}
\frac{d^{2}x}{dz^{2}} + x \left( \frac{1}{\rho^{2}} - k_{f}\right)  = \frac{\Delta p}{p}\frac{1}{\rho}
\label{eqn:disp-eqn-motion}
\end{equation}

which leads to a modified solution of the pseudo-harmonic oscillator equation characterised by the dispersion of the lattice $D(s)$, written as

\begin{equation}
x \left( s \right) = \sqrt{\epsilon} \sqrt{\beta \left( s \right) } cos \left( \psi \left( s \right) + \phi \right) + \frac{\Delta p}{p} D \left( s \right).
\end{equation}

This can be seen as another closed orbit solution to the machine lattice. One side effect of this is the slight change in the tune due to the off momentum particles, characterised by a factor known as the chromaticity $\zeta$, related by

\begin{equation}
\frac{\Delta Q}{Q_{0}} = \zeta \frac{\Delta p}{p}
\end{equation}

where $\Delta Q$ is the tune shift from the lattice tune for an on momentum particle $Q_{0}$. The tune working point is an important factor in designing an accelerator, as crossing integer or low order tune fractions (half, third or quarter integer) allows the summation of kicks in the machine driving large oscillations which may ultimately lead to particle loss. 