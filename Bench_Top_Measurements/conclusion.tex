\section{Conclusion}

In this section we have covered in depth the bench-top measurement techniques for low-Q impedances using the coaxial wire technique. We have described the practical details of setting up the measurement bench, as well as the ways of measuring the longitudinal and transverse impedances for symmetric and asymmetric structures using one and two coaxial wires. Two methods of measurement have been described, the classical coaxial wire measurement method, a transmission measurement through the DUT using a matched VNA circuit, and the resonant coaxial wire measurement method, a transmission method that involves turning the DUT into a coxial resonator, providing a highly sensitive measurement method but at the expense of reduced frequency resolution of the impedance measurements.

We have described a method of measuring the quadrupolar and constant transverse impedances of asymmetric structures using the coaxial measuring techniques, and subsequently evaluated this technique using simulated measurements of a C-core ferrite kicker in HFSS, showing excellent agreement between the analytical calculations and simulated measurements for the longitudinal (real and imaginary), dipolar (real and imaginary), quadrupolar (real only) and constant (real and imaginary) impedances over a large frequency range (1MHz-10GHz). The discrepency for imaginary component of the quadrupolar impedance is considered and attributed to insufficient simulation accuracy.

Futher comments are made on the suitability of coaxial wire measurements of high-Q impedances (for example cavity modes), showing that they can provide inaccurate measurements of the resonant frequency and Q factors of these impedances below cutoff due to additional propogation losses not present when a traversing charged particle sees the structure.

These methods will be used extensively in the evaluation of the beam coupling impedance of the LHC injection kickers in Chap.~\ref{chap:mki} in order to verify the complex simulation model against measurements of the physical magnet.