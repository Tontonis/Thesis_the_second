\section{Conclusion}

In this section we have introduced the concept of wakefields and impedance, both in the lognitudinal and the transverse planes. We have shown that in the transverse plane the impedance may be represented as that dependent on the displacement of the source particle, that dependent on the displacement of the witness particle, and a constant term that is indepedent of the displacement of both. We have briefly covered instabilities that may be driven by beam coupling impedance in the longitudinal and transverse planes. Finally we have studied in depth the phenomenon of beam-induced heating, how it varies depending on the nature of the beam coupling impedance of the device where the heating occurs, on the lognitudinal profile of the bunches, and on the bunch train structure of the machine. This will be used extensively in later chapters when the model of longitudinal distribution of the bunch is shown to be highly important in determining the estimated heat loss, as is the bunch length used for collisions. Using this introduction to the wakefields and impedance, the methods of which they can be measured with both beam and using bench-top measurements shall be discussed, and the ways in which the impedance can be evaluated using simulation tools described.