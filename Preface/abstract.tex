The phenomenon of wakefields and the corresponding frequency-domain property, beam coupling impedance have long been studied in particle accelerators. They are important as a source of beam instabilities due beam-equipment interactions and placing restrictions on the operating parameters of particle accelerators. As accelerators have pushed towards smaller beam emittances and higher beam currents the importance of controlling the beam impedance has become more important. With the latest high power accelerators, even beam-induced heating due to the beam power loss has become a serious limitation on beam operation.

In this thesis a short review of the existing impedance reduction techniques in use in many particle accelerators is given. The limitations and restrictions of these techniques is also considered, focusing heavily on the usage of damping materials to reduce the Q of cavity resonances. In addition to covering a number of methods of measuring (using benchtop and beam-based measurements) or simulating the beam impedance of devices. In particular the use of the coaxial wire technique - a bench top measurement method for measuring the beam impedance of a device - to measure the longitudinal and transverse (dipolar, quadrupolar and constant) impedancess of symmetric and asymmetric structures. These techniques are subsequently applied to two significant sources of beam impedance within the LHC - the injection kicker magnets (LHC-MKIs) and elements of the collimation upgrades for the LHC. A comparison between simulations and measurements of the impedance of the MKI is presented in addition to an evaluation of alternative beam screen designs in regards to the possible beam-induced heating of the structure. Two facets of the LHC collimation upgrade are investigated - the choice of the jaw material of the phase 2 secondary collimators, expected to be a significant contributor to the LHC transverse impedance budget, and the TCTP collimator for which a full structure simulation is carried out to determine the effectiveness of the impedance reduction system in place.