Wakefields and the corresponding frequency-domain phenomenon beam coupling impedance have been well studied for a number of years as a source of beam instabilities within particle accelerators. With the development of the Large Hadron Collider (LHC) and the large beam currents stored in the LHC during fills for physics production, wakefield driven instabilities and strong beam induced heating have become a limiting factors in luminosity production due to both instantaneous luminousity and the available time for collisions.

In this thesis is presented an in depth study of the beam coupling impedance of two important (from both an impedance and operational point of view) pieces of equipment in the LHC; the collimation system and the injection kicker magnets (MKIs). These systems have both been sources of concern for the beam impedance of the LHC, the collimators due to their large transverse impedance and the MKIs due to the strong heating observed during the systematic increase of beam current during operation in 2011 and 2012. The source of the heating for the MKIs is studied in depth, found to be power lost by the beam to wakefields in the MKIs. Simulations and measurements are used to characterise the impedance and localise the components responsible for the high impedance, here the beam screen of the magnet, for which improvements are proposed and verified. A new RF damping system using ferrite for the collimation system is studied and compared to the existing RF damping system, focusing on the heating of the damping system. Highlights include a new method for measuring the quadrupolar and constant transverse impedances of an asymmetric structure using a coaxial wire technique is proposed and verified using computational simulations, and a study of the heat loss in a ferrite damped cavity, focusing on the location of the power loss for cavities being damped to varying degrees. 