% !TEX TS-program = pdflatex
% !TEX encoding = UTF-8 Unicode

% This is a simple template for a LaTeX document using the "article" class.
% See "book", "report", "letter" for other types of document.

\documentclass[12pt]{article} % use larger type; default would be 10pt

\usepackage[utf8]{inputenc} % set input encoding (not needed with XeLaTeX)

%%% Examples of Article customizations
% These packages are optional, depending whether you want the features they provide.
% See the LaTeX Companion or other references for full information.

%%% PAGE DIMENSIONS
\usepackage{geometry} % to change the page dimensions
\geometry{a4paper} % or letterpaper (US) or a5paper or....
% \geometry{margin=2in} % for example, change the margins to 2 inches all round
% \geometry{landscape} % set up the page for landscape
%   read geometry.pdf for detailed page layout information

\usepackage{graphicx} % support the \includegraphics command and options

% \usepackage[parfill]{parskip} % Activate to begin paragraphs with an empty line rather than an indent

%%% PACKAGES
\usepackage{booktabs} % for much better looking tables
\usepackage{array} % for better arrays (eg matrices) in maths
\usepackage{paralist} % very flexible & customisable lists (eg. enumerate/itemize, etc.)
\usepackage{verbatim} % adds environment for commenting out blocks of text & for better verbatim
\usepackage{subfig} % make it possible to include more than one captioned figure/table in a single float
% These packages are all incorporated in the memoir class to one degree or another...

%%% HEADERS & FOOTERS
\usepackage{fancyhdr} % This should be set AFTER setting up the page geometry
\pagestyle{fancy} % options: empty , plain , fancy
\renewcommand{\headrulewidth}{0pt} % customise the layout...
\lhead{}\chead{}\rhead{}
\lfoot{}\cfoot{\thepage}\rfoot{}

%%% SECTION TITLE APPEARANCE
\usepackage{sectsty}
\allsectionsfont{\sffamily\mdseries\upshape} % (See the fntguide.pdf for font help)
% (This matches ConTeXt defaults)

%%% ToC (table of contents) APPEARANCE
\usepackage[nottoc,notlof,notlot]{tocbibind} % Put the bibliography in the ToC
\usepackage[titles,subfigure]{tocloft} % Alter the style of the Table of Contents
\renewcommand{\cftsecfont}{\rmfamily\mdseries\upshape}
\renewcommand{\cftsecpagefont}{\rmfamily\mdseries\upshape} % No bold!

%%% END Article customizations

%%% The "real" document content comes below...

\title{Abstract}
\author{Hugo Day}
%\date{} % Activate to display a given date or no date (if empty),
         % otherwise the current date is printed 

\begin{document}
\maketitle



Wakefields and the corresponding frequency-domain phenomenon beam coupling impedance have been well studied for a number of years as a source of beam instabilities within particle accelerators. With the development of the Large Hadron Collider (LHC) and the large and growing beam currents stored in the LHC during fills for physics production, wakefield driven instabilities and strong beam induced heating in many pieces of equipment have become a limiting factor in luminousity production due to both instantaneous luminousity and the available running time for collisions.

In this thesis is presented an in depth study of the beam coupling impedance of two important (from both an impedance and operational point of view) pieces of equipment in the LHC; the collimation system and the injection kicker magnets (MKIs). These two systems have both been signficant sources of concern for the beam impedance of the LHC, the collimators due to their large transverse impedance and the MKIs due to the large quantity of heating observed during the systematic increase of beam current during operation in 2011 and 2012. The source of the heating for the MKIs is studied in depth, tracing the source to beam-induced heating from power lost by the beam to wakefields in the MKIs. Simulations and measurements are used to characterise the impedance and localise the components responsible for the high impedance, found to be the beam screen of the magnet. Improvements to the beam screen have been proposed and examined from the beam impedance perspective. As part of the verification of the simulation models by comparison of measurement and simulation results, a new method for measuring the quadrupolar and constant transverse impedances of an asymmetric structure using a coaxial wire technique is proposed and verified using computational simulations. For the collimation system a new RF damping system using ferrites to damp cavity modes is studied and compared to the existing RF damping system, again with a focus on the beam-induced heating on the ferrite in the damping system. As part of this, a study of the heat loss within a ferrite damped cavity is presented, focusing on the location of the power loss for cavities being damped to varying degrees. Supporting background information on the calculation of beam-induced heating due to various types of impedances, the use of bench top and beam based measurement of the beam coupling impedance of accelerator components and the machine as a whole and the various types of impedance reduction techniques is also included, in particular the appropriate use of each type of impedance reduction along with the potential drawbacks that they may have in their use.

\end{document}
