\section{Longitudinal Beam Impedance Measurements}

\begin{itemize}
\item{Note we have two impedance components to measure - the real component and the imaginary component}
\item{They both interact with the beam in different ways}
\begin{enumerate}
\item{Real component is directly proportional to the energy lost by a transiting beam}
\item{Imaginary component results in some perturbation of the beam structure}
\end{enumerate}
\end{itemize}

\subsection{Quadrupole Synchrotron Frequency shift}

\subsection{Potential Well Distortion with Bunch Intensity}



\subsection{Synchronous Phase Shift}
\begin{itemize}
\item{}
\end{itemize}

\section{Transverse Beam Impedance Measurements}

For a bunch interacting with a transverse impedance there are two commonly used methods of measuring the transverse impedance - the variance of the coherent betatron tune shift with bunch intensity and the change of growth/decay rate of head tail instabilities with the chromaticity of the beam [ref Sacherer, Zotter].

When a bunch is exposed to a tranverse inmpedance $Z_{\perpendicular}$, it undergoes a complex frequency shift in in betatron frequency

\begin{equation}
\Delta{}\omega_{\beta} = \frac{Nec}{4\sqrt{\pi}\omega_{\beta} \left( E/e \right)T_{0}\sigma_{t}} i\left( Z_{\perpendicular} \right)_{eff}
\end{equation}

where $N$ is the number of particles in the bunch, $E$ is the energy, $T_{0}$ is the revolution frequency, $\sigma_{t} = \sigma_{z}/c$ is the bunch length, $\omega_{\beta} = 2\pi{}Q f_{rev}$ is the betatron frequency and $\left( Z_{\perpendicular} \right)_{eff}$ is the effective transverse impedance. This is given by the impedance convolved with a weighting function h which is determined by the longitudinal bunch profile given by

\begin{equation}
\left( Z_{\perpendicular} \right)_{eff} \left( \omega_{\xhi} \right) = 
\end{equation}

\subsection{Tune shift change with bunch intensity}

\subsection{Growth time change with chromaticity}