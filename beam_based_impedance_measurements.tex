\label{chap:BBmeas}

Beam based measurements are an incredibly useful measurement technique for beam coupling impedance studies. The direct measurements of impedance (either device specific or of the entire machine) provides a reliable comparison to bench top measurements, simulations and analytical calculations. 

There are a number of measurement methods for both the longitudinal and transverse planes using the RF systems and beam instrumentation commonly placed in accelerators. A short introduction to some of the most commonly used of these methods is given here.

\section{Longitudinal Beam Impedance Measurements}

To understand the longitudinal impedance measurements a short introduction to longitudinal motion is given in App.~\ref{app:long_motion}. Here is introduced some of the basics of longitudinal motion of a charged particle within an RF bucket.

\subsection{Potential Well Distortion with Bunch Intensity}
\label{sec:pot-well-dist}

The equation for the longitudinal motion of a charged particle in an RF bucket can be linearised (assuming a small amplitude of the movement within the bucket) and solved to find the so called incoherent synchrotron angular frequency, that is its oscillation in phase space, given by,

\begin{equation}
\omega_{s} = \frac{\omega_{RF}}{\beta} \left( \frac{\eta V_{T}cos\bar{\phi}}{2\phi E/e}  \right)^{\frac{1}{2}}
\label{eqn:syn_volt}
\end{equation}

where $\omega_{RF} = h\omega_{0}$ is the RF frequency, $\omega_{0}$ is the revolution frequency, $h$ is the harmonic
number of the RF, $\eta = \alpha_{c} - \gamma^{-2}$ is the slip factor, $\alpha_{c} = \gamma_{t}^{-2}$ is the momentum compaction factor, $\gamma_{t}$ is the transition energy gamma factor, $E=\gamma E_{0}$ is the total particle energy, $E_{0}$ is the rest energy of the particle, $\bar{\phi}$ is the stable phase angle and $V_{T}$ is the total voltage seen by the beam. For an empty bucket (i.e. no particles present) the voltage present is just that of the RF voltage $V_{RF}$, giving an associated synchrotron frequency $\omega_{s0}$. We can subsequently see from Eqn.~\ref{eqn:syn_volt} that we have the ratio

\begin{equation}
V_{T} = V_{RF}\left(  \frac{\omega_{s}}{\omega_{s0}} \right)^{2}.
\label{eqn:volt_emit}
\end{equation}

Additionally the bunch length $L$ and the energy spread of the bunch $\Delta E/E$ can be shown to be related by

\begin{equation}
L = \frac{\left| \eta \right|c}{\beta\omega_{s}}\frac{\Delta E}{E}
\end{equation}

within the linear approximation. If we assume that the emittance of the bunch is preserved (usually the case in a proton machine), i.e. $L\Delta E = constant$, Eqn.~\ref{eqn:volt_emit} leads to

\begin{equation}
L^{2}\omega_{s} = L_{0}^{2}\omega_{s0}.
\end{equation}

The resulting induced voltage of the bunch depends heavily on the nature of the impedance on which the beam interacts with. In this case we shall illustrate the characteristics of the bunch lengthening with a simple inductive impedance ($Z = j\omega L$). The voltage induced by a bunch with a parabolic line density over a purely inductive impedance, results in a synchrotron frequency given by

\begin{equation}
\omega_{s}^{2} = \omega_{s0}^{2}\left[ 1 - \frac{C}{B^{3}} \frac{I_{b}}{hV_{RF}cos\bar{\phi}} \left( \left| \frac{Z}{n}  \right| - \frac{gZ_{0}}{2\beta\gamma^{2}}   \right)    \right]
\end{equation}

where $c = \frac{3}{\pi^{2}}$ is a constant dependent on the bunch profile, $B = f_{0}2L$ is the bunching
factor, given by the product of the revolution frequency and the bunch length in seconds (or alternatively the ratio of the bunch length to the revolution time), $I_{b}$ is the bunch current, $g = 1 + 2ln\left(\frac{b}{a} \right)$ is the direct space charge factor, $b$ is the beam pipe radius, $a$ is the beam radius, and $Z_{0}$ ($377\Omega$) is the impedance of free space.

If we consider an ultrarelativistic beam ($\gamma \rightarrow \infty$), the space charge factor $g = 0$. Then, considering the value of $\bar{\phi}$ (i.e. whether the bunch is above transition $cos\bar{\phi} > 0$, or below transition $cos\bar{\phi} < 0$), the bunch length increases with the larger bunch current above transition, and decreases below transition. All variables other than $Z/n$ may be measured directly using various instrumentation.

Generalisation of this method to a general complex impedance shows that it is the imaginary component of the longitudinal impedance that contributes to the induced voltage, thus being the component of the impedance measured using this method.

\subsection{Synchronous Phase Shift}
\label{sec:syn-phase-shift}

As discussed in Sec.~\ref{sec:beam_induced_heating}, a particle circulating in a circular accelerator losses energy due to the interaction with impedances in the machine. If the particle is not undergoing acceleration, it can be seen that to keep the same energy, the particle must have a phase relative to the RF that ensures that the energy losses due to wakefields are equal to the energy gained during the traversal of the RF cavities, given by the equivalence of the energy gain by transit and the energy loss by traversal,

\begin{equation}
U = qeVsin\phi
\end{equation}

where $U$ is the total energy lost by a particle per turn, $q$ is the unit charge, $e$ is the electron charge, $V$ is the RF voltage amplitude and $\phi$ is the synchronous phase. The total energy lost is the sum of all energy losses in the machine, given by

\begin{align}
U & = -e^{2}N_{b}k \\
   & = -e^{2}N_{b}\displaystyle\sum\limits_{n} k_{n} \\
   & = -e^{2}N_{b}\displaystyle\sum\limits_{n} \Re{}e\left( Z_{\parallel,n}\left( \omega \right) \right)\left| \lambda \left( \omega \right) \right|^{2} d\omega
\end{align}

where $k_{n}$ is the kick factor of device $n$ in the accelerator and $ Z_{\parallel,n}$ is the longitudinal impedance of device $n$. It can be seen that this allows the measurement of the impedance of the whole device. In addition where the movement of a device is permissible and can be carried out during operation (in collimation systems or insertion devices for example) it is possible to determine the impedance of specific devices by the change in their contribution to the total loss factor. Examples of this measurement method can be found in \cite{Bohl:SingleBunchEnLoss, Argyropoulos:longImpInj}

\section{Transverse Beam Impedance Measurements}


For a bunch interacting with a transverse impedance there are two commonly used methods of measuring the transverse impedance - the variance of the coherent betatron tune shift with bunch intensity and the change of growth/decay rate of head-tail instabilities with the chromaticity of the beam \cite{Sacherer:BunchBeamEffects}.

When a bunch is exposed to a tranverse dipolar impedance $Z_{\perp}$, it undergoes a complex frequency shift in betatron frequency


\begin{equation}
\Delta{}\omega_{\beta} = \frac{N_{p}ec}{4\sqrt{\pi}\omega_{\beta} \left( E/e \right)T_{0}\sigma_{t}} j\left( Z_{\perp} \right)_{eff}
\label{eqn:complex_tun_shift}
\end{equation}

where $N_{p}$ is the number of particles in the bunch, $E$ is the energy, $T_{0}$ is the revolution time, $\sigma_{t} = \sigma_{z}/c$ is the bunch length, $\omega_{\beta} = 2\pi{}Q f_{rev}$ is the betatron frequency and $\left( Z_{\perp} \right)_{eff}$ is the effective transverse impedance. This is given by the impedance convolved with a weighting function $h$ which is determined by the transverse bunch profile

\begin{equation}
\left( Z_{\perp} \right)_{eff} \left( \omega_{\xi} \right) = \int_{-\infty}^{\infty} Z_{\perp} \left( \omega \right) h_{m} \left(  \omega - \omega_{\xi} \right) d\omega.
\end{equation}

As an example, for the 0-mode coherent bunch oscillation and assuming a gaussian bunch profile the weighting function can be written as

\begin{equation}
h_{0} \left( \omega \right) = \frac{\sigma_{t}}{\sqrt{\pi}}e^{ \left(  \omega \sigma_{t}  \right)^{2}}.
\label{eqn:weighting_func}
\end{equation}

It can be seen from Eqn.~\ref{eqn:weighting_func} that the function is centred on $\omega_{\xi}$ (the chromatic frequency) which is dependent on the chromaticity $\xi$ and the phase slip factor $\eta$ as

\begin{equation}
\omega_{\xi} = \xi \frac{\omega_{\beta}}{\eta}.
\end{equation}

Eqn~\ref{eqn:complex_tun_shift} indicates that the imaginary component of the effective transverse impedance contributes to a real coherent tune shift, and the real component to an imaginary tune shift, visible as a growth/decay time in the oscillation. For a broadband impedance ($Q=1$), at low frequencies it is possible to assume that $\Im{}m\left(  Z_{\perp} \right) \approx const.$ It then follows that this is directly proportional to the real tune shift which can be obtained by measuring the coherent tune as a function of bunch intensity.

This can be thought by considering the complex tune shift ($\Delta \omega_{\beta} = \omega_{\beta} + j\omega_{i}$) and considering the change to the solution to the transverse equation of motion

\begin{align}
e^{j\left( \omega + \Delta \omega \right) t } &= e^{j\left( \omega + \omega_{\beta} + j \omega_{i} \right) t } \nonumber \\
& = e^{j\left( \omega + \omega_{\beta}\right) t }e^{- \omega_{i}  t } \nonumber \\
& = e^{j\left( \omega + \omega_{\beta}\right) t }e^{ t_{g} }
\end{align}

where $t_{g} = f ( \Im{}m { (  Z_{\perp} )_{eff} } )$ is the time constant.
 
\subsection{Tune shift change with bunch intensity}
\label{sec:tune-shift-bunch-int}

The real component of the solution to Eqn~\ref{eqn:complex_tun_shift} can be related to the integer betatron tune by

\begin{equation}
\Delta Q = \frac{\Omega - \omega_{\beta}}{\omega_{0}} \approx \frac{1}{\omega_{0}} \frac{N_{p}ec^{2}}{4\sqrt{\pi}\omega_{\beta} \left( E/e \right)T_{0}\sigma_{t}} \Im{}m\left(  Z_{\perp} \right)_{eff},
\end{equation} 

where $\Omega$ is the measured betatron frequency. It can be seen that, by altering the bunch population, the tune shift $\Delta Q$ can be measured and subsequently the value for $\Im{}m\left(  Z_{\perp} \right)_{eff} = Z_{\perp}^{dip} + Z_{\perp}^{quad}$ can be obtained.

\subsection{Growth time change with chromaticity}
\label{sec:growth-time-chrom}

Similarly to the tune shift measurements, we consider the imaginary component of the solution. Similarly to a harmonic oscillator, the imaginary part of the solution denotes a growing/damping time of the oscillation, $\tau$, in this case given by

\begin{equation}
\frac{1}{\tau} = -T_{0} \frac{N_{p}ec^{2}}{4\sqrt{\pi}\omega_{\beta} \left( E/e \right)T_{0}\sigma_{t}} \Re{}e\left(  Z_{\perp} \right)_{eff}.
\end{equation}

$\Re{}e\left(  Z_{\perp} \right)_{eff}$ is dependent only on the chromaticity $\xi$, being zero at $\xi = 0$ and different from zero for other values of chromaticity. Depending on whether the measurements are carried out above or below transition this produces either a damping for positive chromaticity above transition and growth below transition, and vice versa for negative chromaticity.

\section{Conclusion}

In this chapter we have introduced ways in which it is possible to measure the longitudinal and transverse impedance (real and imaginary components) of circular accelerators. This has been done to give an awareness of the ways in which it is possible to measure either the impedance of the machine as a whole, or in the case of devices that are moveable, their contribution to the machine impedance. Not all devices are suitable to bench-top meausrements, either due to structure (long, heavy, beam path not condusive to bench top measurements), safety concerns (maybe radioactive) or there not being sufficient time to carry out the measurements, thus beam based measurements are an exceptionally powerful tool in impedance evaluation.