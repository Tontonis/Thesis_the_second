\section{Conclusion}

In this chapter we have evaluated in depth the longitudinal impedance due to resistive wall and cavity modes in the TCTP collimator, a tertiary collimator to be placed in the LHC. The beam induced heating expected in this collimator has been evaluated, significant as it is the first collimator to be installed in the LHC with a new RF system, designed to damp cavity modes by the use of RF tiles rather than screen the surrounding volume by the use of transverse RF fingers, done as the RF fingers were suspected to be a cause of dust in the LHC. It has been found that the placement of the ferrite tiles was not ideal from the point of view of damping the cavity modes, and as such the Q of the cavity modes is not as reduced as could be the case. However studies of the thermal evaluation of the ferrite and the collimator indicate that the layout is compatible with the thermo-mechanical requirements of the structure, and the present design is well understood and thus likely to be kept. The proposed ferrite has some vacuum issues, and alternative damping materials are being investigated.

In addition a small study of the possible materials for the phase 2 secondary collimator jaw material is carried, with a recommendation for a good conducting material given, either copper or molybdenum of the possibilities given.