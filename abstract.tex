Wakefields and the corresponding frequency-domain phenomenon, beam coupling impedance, have been well studied for a number of years as a source of beam instabilities within particle accelerators. With the development of the Large Hadron Collider (LHC) and the large and growing beam currents stored in the LHC during fills for physics production, wakefield driven instabilities and strong beam induced heating in many pieces of equipment have become a limiting factor in luminosity production due to both instantaneous luminosity and the available running time for collisions.

In this thesis an in depth study of the beam coupling impedance of two important (from both an impedance and operational point of view) pieces of equipment in the LHC; the collimation system and the injection kicker magnets (MKIs) is presented. These two systems have both been signficant sources of concern for the beam impedance of the LHC, the collimators due to their large transverse impedance and the MKIs due to the large quantity of heating observed during the systematic increase of beam current during operation in 2011 and 2012. The source of the heating for the MKIs is studied in depth, tracing the source to beam-induced heating from power lost by the beam to wakefields in the MKIs. Simulations and measurements are used to characterise the impedance and localise the components associated for the high impedance, found to be the ferrite yoke not being completely screened by the beam screen of the magnet. Improvements to the beam screen have been proposed and examined from the beam impedance perspective and validated after installation in the LHC. As part of the verification of the simulation models by comparison of measurement and simulation results, a new method for measuring the quadrupolar and constant transverse impedances of an asymmetric structure using a coaxial wire technique is proposed and verified using computational simulations. Based on these simulations and new MKI (MKI8d) was built and installed in the LHC drastically reducing the operating temperature of this magnet. For the collimation system a new RF damping system using ferrites to damp cavity modes is studied and compared to the existing RF damping system, again with a focus on the beam-induced heating on the ferrite in the damping system. As part of this, a study of the heat loss within a ferrite damped cavity is presented, focusing on the location of the power loss for cavities being damped to varying degrees. Supporting background information on the calculation of beam-induced heating due to various types of impedances, the use of bench top and beam based measurement of the beam coupling impedance of accelerator components and the machine as a whole and the various types of impedance reduction techniques is also included, in particular the appropriate use of each type of impedance reduction along with the potential drawbacks that they may have in their use.