\section{Conclusion}

In this section with have given a thorough review of the beam coupling impedance reduction measures that are presently in use in many particle accelerators. We have described in detail the use of each reduction method, giving notes as to the side-effects each may have, how they may tuned to give the desired impedance response and some of the limits on their implementation. Some examples of the successful implementation of these impedance reduction measures on real devices is given for clarity.

In addition an indepth study of the location of power loss due to EM fields in the case of ferrite damped cavity structures is presented. It has been found that for the case of mildly damped structures (either due to a weakly damping material or placement of the ferrite in a location where it does not interact strongly with the cavity fields) that the percentage of the power loss that occurs within the ferrite is linear with the reduction of the Q-factor of the cavity mode (i.e. percentage of power lost in the ferrite is negatively correlated with the Q-factor). The absolute value of power loss reaches a peak value for a very low degree of damping, upon increasing the damping strongly decreases the power loss in the damping material. This has given a better understanding of how the power loss in a structure changes as it's damping is modified, essential for the correct treatment of the thermal evolution of the structure under heat load.

These impedance reduction techniques are now implemented in the context of the LHC collimators and the injection kicker magnets, where their usefulness will discussed, and the other design restrictions that might limit their performance become more prominant in decision making.