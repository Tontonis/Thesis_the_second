\section{Beam screens in Kicker Magnets}
\label{sec:beam_screens}

A substantial contributor to the beam coupling impedance in many machines, in both the longitudinal and transverse planes, are kicker magnets. These are magnets that generate a pulsed magnetic field for a limited period of time (i.e. not always on during beam operation as the dipoles and quadrupoles used for beam optics are), often for orbit corrections, extraction and injection. They have been known to be a problematic component of particle accelerators for a number of decades, mostly in lepton machines due to the traditionally higher beam currents in these machines.

In lepton machines kicker magnets can suffer from problems of heating due to eddy currents induced by traversing bunches, and the neccessity that a remedy to this solution maintain the rapid rise time of the magnetic field required for normal kicker magnet operation, which typically requires a field rise time on the order of the bunch or bunch train seperation in a machine \cite{Caspers:ThinCondLayers}. 

In the LHC gaps are deliberately left between batches of beams in order to allow for the rise/fall of the magnetic field of kicker magnets. Thus in order to maximise the integrated beam intensity, the rise and fall time must be kept to a minimum, bearing in mind technical and economic constraints: for the LHC injection kickers the field rise and fall times are 900ns and 3$\mu$s respectively.

As mentioned above, kicker magnets contribute a substantial fraction of the beam coupling impedance of an accelerator. In order to reduce the beam coupling impedance of a kicker magnet, a thin conductive coating or longitudinal conductors can be incorporated between the beam and the yoke of the kicker magnet to shield the yoke against the beam image current. However any such beam impedance reduction measures must not significantly increas the field rise and fall time of the kicker magnet.

For the LHC injection system a ceramic chamber is incorporated in th aperature of each kicker magnet: the purpose of the ceramic chamber is to mechanically support the conductors which shield the yoke against the beam image current. In addition the ceramic chamber electrically isolates the shield from the high voltage and return busbar of the kicker magnets.

However the ceramic chamber contributes a large imaginary component to the longitudinal impedance, due to its high permitivitty and poor conductivity. In addition it is liable to build up static charges due to being an insulator in a region subject to high electric and magnetic fields.

Shielding will be studied in further depth in Chap.~\ref{chap:mki}, with particular attention to the limitations in use in a high current hadron machine.
