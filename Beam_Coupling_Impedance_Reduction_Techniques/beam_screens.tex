\section{Beam screens in kicker magnets}
\label{sec:beam_screens}

A substantial contributor to the beam coupling impedance in many machines, in both the longitudinal and transverse planes, are kicker magnets. These are magnets that generate a pulsed magnetic field for a limited period of time (i.e. not always on during beam operation as the dipoles and quadrupoles used for beam optics are), often for orbit corrections, extraction and injection. They have been known to be a problematic component of particle accelerators for a number of decades, mostly in lepton machines due to the traditionally higher beam currents that operate in these machines.

In lepton machines they suffered from problems of heating due to eddy currents induced by traversing bunches, and the neccessity that a remedy to this solution maintain the rapid rise time of the magnetic field required for normal kicker magnet operation, which typically requires a field rise time on the order of the bunch or bunch train seperation in a machine [citing all the cite]. 

A plain ceramic chamber contributes it's own problem from a beam impedance point of view, especially contributing a large imaginary component to the longitudinal impedance due to it's high permitivitty and poor conductivity. In addition it is liable to build up static charges due to being an insulator in a region subject to high electric and magnetic fields. The solution to these problems neccessitates a thin conductive coating on the inside surface of the chamber, either a continuous thin layer (which itself can greatly reduce the field rise time of the kicker magnetic field), or the use of longitudinal stripes, which carry a large proportion of the beam image current, whilst maintaining the rise time characteristics of the magnetic field.

This method will be studied in further depth in Sec.~\ref{sec:mki_studies}, with particular attention to the limitations in use in a high current hadron machine.
