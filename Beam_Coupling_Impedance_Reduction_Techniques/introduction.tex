Due the effects of wakefields on both beam stability and machine equipment it is often necessary to consider reducing the beam impedance of different components in particle accelerators. There are a number of ways in which it is possible to reduce the impedance depending on whether the impedance is primarily geometric or material dependent in nature and the most commonly used will be reviewed. Different solutions from mechanical changes in the structures to damping materials placed to damp resonances are discussed. Further references are given to provide more in depth knowledge as required. In particular the use of ferrite to damp cavity modes that may not be removed by redesign, due to either time or mechanical constraints, has recently become a product of intensive study due to the high temperatures seen in many devices that have ferrite placed in them, with concern that the ferrites may heat beyond their Curie temperature during regular operation leading to a deteriorating case for the machine impedance. 