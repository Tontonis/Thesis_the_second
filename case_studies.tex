
\section{LHC Injection Kicker Magnet}

\begin{itemize}
\item{Introduction to the kicker magnet system}
\begin{enumerate}
\item{What are kicker magnets - Injection/Extractions systems}
\item{Why are they potentially a problem}
\end{enumerate}
\item{Explain the background of the LHC-MKI in particular}
\begin{enumerate}
\item{The original concern over heating, subsequent design of the beam screen}
\item{Observed problems with electrical breakdown of the beam screen, and subsequent removal of screen conductors}
\item{Recent observed heating in MKIs}
\end{enumerate}
\item{Summarise current state of the MKIs in the LHC - Beam screen layouts, two sets of kickers, one all of 15 screen conductors, one has one with 24}
\item{Comparison of the measurements and simulations of the LHC-MKI}
\begin{enumerate}
\item{Measurements of the Longitudinal BCI of the MKI - before and after bake out, with 15 and 19 screen conductors}
\item{Measurements of the transverse BCI of the MKI - if time just for interest and as a verification of the asymmetric method}
\end{enumerate}
\item{A breakdown of the impedance that we see in the MKI}
\begin{enumerate}
\item{Start with a simple c - core ferrite magnet}
\item{Add a ceramic tube}
\item{Add screen conductors in internal side - Brief interlude about the limitations this places on the magnet rise time due to creating a Faraday cage}
\item{Add the capacitive coupling - Different lengths of overlap to demonstrate that this controls the frequency of the resonances. Also lengths of the screen conductors for lower resonances}
\item{Add the ferrite damping rings - damp resonances of length of screen conductor - not(!) overlap}
\item{Hopefully show that this is the dominant cause of the resonances}
\end{enumerate}
\item{Summary of different beam screen designs - Where possible include discussion about the reduction of the voltage build up on each screen conductor}
\begin{enumerate}
\item{Screen conductors all of the same length with capactive coupling at one end - Show how increasing the number of screen conductors really helps to reduce the BCI}
\item{Screen conductors with a tapering of the length, with the longest at the side towards the ground plate and the shortest towards the HV plate}
\item{Alternating lengths of long and short screen conductors}
\item{Having the screen conductors in closed slots in the ceramic tube}
\item{The addition of small conducting spheres to the ends of the screen conductors to reduce the high fields at the conductor ends}
\item{Thicker ceramic at the capacitively coupled end of the beam screen to reduce the field gradient}
\item{Alternative beam screen design - Most screen conductor capactively coupled at both ends, with two connected to the beam pipe at one end. Aim to reduce the potential on all conductors by conductively connecting them at the capactively coupled end}
\item{Stepping the external metallization away from the ceramic tube at the ends of the screen conductors. The metallization will be removed and a conducting pipe placed there instead - different step out dstances are investigated}
\end{enumerate}

\item{Heating estimates for all of the above}
\begin{enumerate}
\item{Explain completely the methods of estimating the power losses here - bunch intensity, number of bunches, bunch length, distribution}
\item{Note the benefits of increasing the bunch length for the resonances with 15 screen conductors}
\item{Summary charts of the beam induced heating for the others, and plots illustrating how the changes in bunch length changes the power loss}
\item{Impedance profiles of all of the above - longitudinal predominantly}
\item{Some judgement on which is most appropriate for an impedance point of view}
\item{Comments on the improvements made to existing magnets already - 19 screen conductors}
\end{enumerate}
\end{itemize}
%
% Introduction to kicker magnet systems - PFNs and the materials normally used
% Risks of beam coupling impedance to the kicker magnet system
% Existing beam coupling impedance reduction - ceramic screen with conductive inserts. Limits of this (electrical breakdown)
% First - comparison of simulations to measurements
% Second - Proposals of improvements to the reduction measures - Rounded ends, changing strips arrangements
% Evaluation of new proposals - comparison of heat load (different methods of evaluating heat load (assume crosses resonance, or real spectrum)), explaination of 
% spectrum measurements and the differences bunch length and profile can make
%
%
%



\section{LHC Phase 2 Collimator Designs}
\begin{itemize}
\item{Introduction to the collimator upgrade project - Why are collimators important}
\begin{enumerate}
\item{The have two significant physical requirements - a rigid, sturdy material and must be placed very close to the beam}
\item{The first necessitated the use of graphite/carbon materials for the phase 1 collimators due to their survivability in the condition needed. The second means that the resistive wall impedance is very large}
\end{enumerate}
\item{Phase 2 collimator materials choice}
\begin{enumerate}
\item{Why a phase 2 collimator upgrade?}
\item{Summary of the material requirements and the available materials}
\item{Simple model used - simulations and comparison to analytical models}
\end{enumerate}
\item{TCTP Impedance Studies}
\begin{enumerate}
\item{What is the TCTP? Why do we need it?}
\item{Possibly designs}
\begin{itemize}
\item{Structure with RF fingers isolating the beam from the vacuum tank}
\item{Structure with a narrow connection between central cavity and vacuum tank - simulated with and without ferrite to demonstrate reduction in Q}
\end{itemize}
\item{Simulations Parameters and results}
\item{Heating estimates - using single bunch, multi-bunch, on resonance and equally spaced}
\item{Localisation of the heating - using both CST and HFSS}

\end{enumerate}
\end{itemize}
%
% Introduction to the new collimator design - Old collimator design, problems with sliding rails, new design, new material possibilities
% Material Evaluation - Comparison of CST simulations
% Whole collimator simulations of phase 2 secondary - Transverse and longitudinal modes with ferrite - compared to that from phase 1
% Sims in CST PS, GdFidl and HFSS
% Heating calculations also
%
%
%
%
%


