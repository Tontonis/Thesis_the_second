\section{Simulations and Measurements of the MKI with 15 screen conductors}

To begin an analysis of the heating of the MKIs, and to verify the use of simulations as a tool for evaluating different designs of the MKI impedance reduction methods it was important to measure an existing MKI configuration as a benchmark. Due to the expected low beam coupling impedance of the MKI even with 15 screen conductors (on the order of 100$\Omega$ longitudinal beam coupling impedance) it was decided to do the measurements using the resonant coaxial wire method, described in Sec.~\ref{sec:reson-coax-meth}. This method limits the frequency resolution of the measurements ($\Dellta{}f = c/\Delta \lambda = c/2L_{device} \approx 40MHz$ where $c$ is the speed of light and $L_{device}\approx 3.5m$ is the length of the MKI in it's vacuum tank), but the added accuracy in this measurement is considered appropriate in this case.

The measurements were carried out on a fully assembled MKI inside it's vacuum tank, with a beam screen housing 15 tapered screen conductors. For the single wire measurements a copper wire of radius 0.5mm was placed in the central beam pipe, suspended on a vacuum flange modified to take a Sucobox connection allowing a capacitor housed in a seperate sucobox to be attached to the connecting box (The experimental setup is shown in Fig.~\ref{fig:mki-meas-setup}). To displace the wire for displaced single wire measurements, a pair of nylon screws (one at each flange) were used to physically displace the wire. Displacements were taken at 3mm intervals in both the vertical and horizontal planes, between -9mm and +9mm displacement.

For the two wire measurements, two wires were suspended in the device, seperated by 7mm. The wires were connected to a single VNA, each connection run through a 180$^{\circ}$ hybrid to generate the appropriate phase difference between the two. The experimental setup for the two wire measurements is shown in Fig.~\ref{fig:mki-meas-two-wire}.

The VNA is calibrated using an 8532E calibration kit, using an IF bandwidth of 1kHz. 2000 equally space data points are used over a frequency range of 1MHz-2GHz. The Q factors and resonant frequencies of the resonant modes are calculated using a peak finding algorithm on the VNA, and the conductive losses due to the copper wire calculated analytically and compensated for.

\begin{figure}
\subfigure[]{
\label{fig:mki-meas-vac-tank}
}
\subfigure[]{
\label{fig:mki-coupling-capacitor}
}
\label{fig:mki-meas-setup}
\caption{The measurement setup of the coaxial wire measurements of an LHC-MKI. Shown is the vacuum tank with connecting N-type cables \ref{fig:mki-meas-vac-tank} and a coupling capacitor in it's associated Sucobox \ref{fig:mki-coupling-capacitor}.}
\end{figure}

\begin{figure}
\label{fig:mki-meas-two-wire}
\caption{The measurement setup for the two wire measurement setup. The 180$^{\circ}$ hybrid can be seen on top of the vacuum flange.}
\end{figure}

The results for the measurements be seen in the following figures:
\begin{enumerate}
\item{Fig.~\ref{fig:mki-15-longitudinal} the longitudinal impedance}
\item{Fig.~\ref{fig:mki-15-horz-dipolar} the horizontal dipolar impedance}
\item{Fig.~\ref{fig:mki-15-vert-dipolar} the vertical dipolar impedance}
\item{Fig.~\ref{fig:mki-15-horz-quad} the horizontal quadrupolar impedance}
\item{Fig.~\ref{fig:mki-15-vert-quad} the vertical quadrupolar impedance}
\item{Fig.~\ref{fig:mki-15-horz-constant} the horizontal constant impedance}
\item{Fig.~\ref{fig:mki-15-vert-constant} the vertical constant impedance}
\end{enumerate}

\begin{figure}
\label{fig:mki-15-longitudinal}
\caption{The longitudinal impedance of the LHC MKI acquired by measurements using the resonant coaxial wire method and time domain simulations using CST Particle Studio.}
\end{figure}

\begin{figure}
\subfigure[]{
\label{fig:mki-15-horz-dipolar}
}
\subfigure[]{
\label{fig:mki-15-vert-dipolar}
}
\label{fig:mki-15-dipolar}
\caption{The dipolar impedances of the LHC MKI acquired by measurements using the resonant coaxial wire method and time domain simulations using CST Particle Studio. \ref{fig:mki-15-horz-dipolar} shows the horizontal dipolar impedance, and \ref{fig:mki-15-vert-dipolar} the vertical dipolar impedance}
\end{figure}

\begin{figure}
\subfigure[]{
\label{fig:mki-15-horz-quad}
}
\subfigure[]{
\label{fig:mki-15-vert-quad}
}
\label{fig:mki-15-quadrupolar}
\caption{The quadrupolar impedances of the LHC MKI acquired by measurements using the resonant coaxial wire method and time domain simulations using CST Particle Studio. \ref{fig:mki-15-horz-quad} shows the horizontal quadrupolar impedance, and \ref{fig:mki-15-vert-quad} the vertical quadrupolar impedance}
\end{figure}

\begin{figure}
\subfigure[]{
\label{fig:mki-15-horz-constant}
}
\subfigure[]{
\label{fig:mki-15-vert-constant}
}
\label{fig:mki-15-constant}
\caption{The constant transverse impedances of the LHC MKI acquired by measurements using the resonant coaxial wire method and time domain simulations using CST Particle Studio. \ref{fig:mki-15-horz-constant} shows the horizontal constant impedance, and \ref{fig:mki-15-vert-constant} the vertical constant impedance}
\end{figure}
\begin{itemize}
\item{Introduction to the kicker magnet system}                                                                                                                                                                                                                                                                                                             
\begin{enumerate}
\item{What are kicker magnets - Injection/Extractions systems}
\item{Why are they potentially a problem}
\end{enumerate}
\item{Explain the background of the LHC-MKI in particular}
\begin{enumerate}
\item{The original concern over heating, subsequent design of the beam screen}
\item{Observed problems with electrical breakdown of the beam screen, and subsequent removal of screen conductors}
\item{Recent observed heating in MKIs}
\end{enumerate}
\item{Summarise current state of the MKIs in the LHC - Beam screen layouts, two sets of kickers, one all of 15 screen conductors, one has one with 24}
\item{Comparison of the measurements and simulations of the LHC-MKI}
\begin{enumerate}
\item{Measurements of the Longitudinal BCI of the MKI - before and after bake out, with 15 and 19 screen conductors}
\item{Measurements of the transverse BCI of the MKI - if time just for interest and as a verification of the asymmetric method}
\end{enumerate}
\item{A breakdown of the impedance that we see in the MKI}
\begin{enumerate}
\item{Start with a simple c - core ferrite magnet}
\item{Add a ceramic tube}
\item{Add screen conductors in internal side - Brief interlude about the limitations this places on the magnet rise time due to creating a Faraday cage}
\item{Add the capacitive coupling - Different lengths of overlap to demonstrate that this controls the frequency of the resonances. Also lengths of the screen conductors for lower resonances}
\item{Add the ferrite damping rings - damp resonances of length of screen conductor - not(!) overlap}
\item{Hopefully show that this is the dominant cause of the resonances}
\end{enumerate}
\item{Summary of different beam screen designs - Where possible include discussion about the reduction of the voltage build up on each screen conductor}
\begin{enumerate}
\item{Screen conductors all of the same length with capactive coupling at one end - Show how increasing the number of screen conductors really helps to reduce the BCI}
\item{Screen conductors with a tapering of the length, with the longest at the side towards the ground plate and the shortest towards the HV plate}
\item{Alternating lengths of long and short screen conductors}
\item{Having the screen conductors in closed slots in the ceramic tube}
\item{The addition of small conducting spheres to the ends of the screen conductors to reduce the high fields at the conductor ends}
\item{Thicker ceramic at the capacitively coupled end of the beam screen to reduce the field gradient}
\item{Alternative beam screen design - Most screen conductor capactively coupled at both ends, with two connected to the beam pipe at one end. Aim to reduce the potential on all conductors by conductively connecting them at the capactively coupled end}
\item{Stepping the external metallization away from the ceramic tube at the ends of the screen conductors. The metallization will be removed and a conducting pipe placed there instead - different step out dstances are investigated}
\end{enumerate}

\item{Heating estimates for all of the above}
\begin{enumerate}
\item{Explain completely the methods of estimating the power losses here - bunch intensity, number of bunches, bunch length, distribution}
\item{Note the benefits of increasing the bunch length for the resonances with 15 screen conductors}
\item{Summary charts of the beam induced heating for the others, and plots illustrating how the changes in bunch length changes the power loss}
\item{Impedance profiles of all of the above - longitudinal predominantly}
\item{Some judgement on which is most appropriate for an impedance point of view}
\item{Comments on the improvements made to existing magnets already - 19 screen conductors}
\end{enumerate}
\end{itemize}
%
% Introduction to kicker magnet systems - PFNs and the materials normally used
% Risks of beam coupling impedance to the kicker magnet system
% Existing beam coupling impedance reduction - ceramic screen with conductive inserts. Limits of this (electrical breakdown)
% First - comparison of simulations to measurements
% Second - Proposals of improvements to the reduction measures - Rounded ends, changing strips arrangements
% Evaluation of new proposals - comparison of heat load (different methods of evaluating heat load (assume crosses resonance, or real spectrum)), explaination of 
% spectrum measurements and the differences bunch length and profile can make
%
%
%

