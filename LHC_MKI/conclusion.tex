\section{Conclusion}

In this chapter we have introduced the background of the LHC MKIs, of the reasons behind changing the screen conductor layout due to electrical breakdown and of the observed temperature increase during the increase in bunch intensity in the LHC during operation in 2011 and 2012. The impedance of the beam screen with 15 screen conductors has been thoroughly evaluated using both computational simulations and bench-top measurements, which show a good agreement between the two. Based on this confidence in the simulation model, a new design with 19 screen conductors was evaluated, and placed in the LHC (MKI8d) during technical stop 3 in September 2012. Temperatures measured before and after the change show a drastic reduction in the observed maximum temperature reached, demonstrating that the proposed solution was effective in reducing the real component of the longitudinal beam coupling impedance of the MKI. Using this confidence in the simulation tools and the resulting manufactured magnet, a new design that would further reduce the heat load on the magnet, whilst reducing the likelihood of electrical breakdown on the ceramic tube of the beam screen was proposed, the impedance evaluated, and the heat load calculated. This design was shown to further reduce the maximum temperature reached during operation, whilst satisfying electrical breakdown requirements. As a result this design is being pushed for deployment in all 8 MKI magnets. 

Soon electrical breakdown measurements of the new design is to be carried out, along with bench-top impedance measurements using the coaxial wire method. The resulting magnets are intended to placed in the LHC during long shutdown 1.