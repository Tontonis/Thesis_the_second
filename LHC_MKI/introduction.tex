
\begin{itemize}
\item{Introduction to the kicker magnet system}
\begin{enumerate}
\item{What are kicker magnets - Injection/Extractions systems}
\item{Why are they potentially a problem}
\end{enumerate}
\item{Explain the background of the LHC-MKI in particular}
\begin{enumerate}
\item{The original concern over heating, subsequent design of the beam screen}
\item{Observed problems with electrical breakdown of the beam screen, and subsequent removal of screen conductors}
\item{Recent observed heating in MKIs}
\end{enumerate}
\item{Summarise current state of the MKIs in the LHC - Beam screen layouts, two sets of kickers, one all of 15 screen conductors, one has one with 24}
\end{itemize}

The injection kicker magnets in the LHC are part of the injection system for the LHC machine. This system is used to match the trajectory of the injected beam to that of the stable beam path in the accelerator. An example schema for the LHC can be seen in Fig.~\ref{fig:injection-system-schema}. The system typically uses two components, a septum, which may provide a slowly rising and falling time, but strong, field, and kickers, which may provide a rapidly rising and falling, but comparatively weak field, to match the injected beam to the correct trajectory. Similar components are used for the extraction of beam also.

\begin{figure}

\label{fig:injection-system-schema}
\caption{An example layout of a injection system for an accelerator.}
\end{figure}

By design kickers must always be visible to the beam due to the need to quickly fire to apply their kick. In addition the need for a highly homogeneous field whilst the kick is applied, as well as the strength of the field neccesitates that the aperture for kickers often be very narrow, meaning they are in close proximity to the beam. This leads to two concerns - that the close proximity of the beam to a device that may be made of either a highly lossy material [cite ferrite kicker magnets] or of a source of strong geometrical impedance [cite stripline kickers] may be source of impedance that drives instabilities in the beam [impedance of the SPS], and secondly that the large real component of ferrite kicker magnets may be subject to intense heating in high beam current machines [cite beam induced heating in the SPS MKEs]. 