
\begin{itemize}
\item{Introduction to the kicker magnet system}
\begin{enumerate}
\item{What are kicker magnets - Injection/Extractions systems}
\item{Why are they potentially a problem}
\end{enumerate}
\item{Explain the background of the LHC-MKI in particular}
\begin{enumerate}
\item{The original concern over heating, subsequent design of the beam screen}
\item{Observed problems with electrical breakdown of the beam screen, and subsequent removal of screen conductors}
\item{Recent observed heating in MKIs}
\end{enumerate}
\item{Summarise current state of the MKIs in the LHC - Beam screen layouts, two sets of kickers, one all of 15 screen conductors, one has one with 24}
\end{itemize}

The injection kicker magnets in the LHC are part of the injection system for the LHC machine. This system is used to match the trajectory of the injected beam to that of the stable beam path in the accelerator. An example schema for the LHC can be seen in Fig.~\ref{fig:injection-system-schema}. The system typically uses two components, a septum, which may provide a slowly rising and falling time, but strong, field, and kickers, which may provide a rapidly rising and falling, but comparatively weak field, to match the injected beam to the correct trajectory. Similar components are used for the extraction of beam also.

\begin{figure}

\label{fig:injection-system-schema}
\caption{An example layout of a injection system for an accelerator.}
\end{figure}