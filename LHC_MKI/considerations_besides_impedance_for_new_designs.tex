\section{Other Concerns for the MKI Operation and Temperature Reduction}

Given the breadth of concerns that must also be considered alongside beam impedance when discussing the heating of the MKIs and alterations to the beam screen design it is suitable to include a brief discussion on these matters here. 

First of the concerns is the induced electrical potential in the screen conductors during the firing of the kicker magnet. It is due to this high induced voltage (up to 45kV on the conductor closest to the HV busbar) and the resulting sparking/electrical breakdown between screen conductors and to the end of the beam screen at the capacitively coupled end that were the initial motivation to remove the 9 screen conductors from the beam screen. The reasoning for this was that sparking produces vacuum spikes, can cause the decay of the screen condition and may be a source of UFOs (unidentified falling objects - microscopic dust particles that drop through the beam causing spikes in beam losses due to scattering, resulting in the beam being dumped). 

In addition to improving the screening of the ferrite yoke from the beam any proposed beam screen design must also allow the PFN to be fired at the desired voltage (54kV from the technical specifications). Work by Barnes et al \cite{Barnes:eFieldMKI} has found there are a number of factors that can reduce the induced potential on the screen conductors. These include the use of elongated spheres on the ends of the screen conductors to reduce the high electric fields at the ends of the screen conductors (Fig.~\ref{fig:mki-elongated-spheres}), alternating the length of the screen conductors (Fig.~\ref{fig:mki-alternating-length-conds}) and replacing the metallization with a cylindrical metal tube away from the surface near the areas of overlap between the screen conductors and external metallization (Fig.~\ref{fig:mki-external-metallization}). The aim of this last measure is to remove the external ground further away from the outside of the ceramic beam screen, as the metallization has been observed to force the ground plane closer to the screen conductors due to the high permitivitty of the ceramic beam screen. The effects of these changes on the beam coupling impedance are discussed in later sections.

The second major concern is the transfer of thermal energy out of the kicker magnets. Due to the HV pulsed nature of the kicker magnet operation there is no active cooling within the device, thus cooling is reliant exclusively on radiative transfer of the heat from the components to the surrounding air via the vacuum tank. Experiments have shown that the internal surface of the vacuum tank has a very low emissivity in the IR range \cite{Barnes:emisMKI}, and subsequent thermal simulations have shown that the stable temperature of the MKI is very sensitive to the emissivity of the vacuum tank internal surface \cite{Garlasche:2dHeatEmis}. Subsequently a significant quantity of work has been carried out on methods to improve the emissivity of the internal surface of the vacuum tank so as to more effectively passively cool the device.