\section{Other Concerns for the MKI Operation and Temperature Reduction}

Given the breadth of concerns that must also be considered alongside beam impedance when discussing the heating of the MKIs and alterations to the beam screen design it is suitable to include a brief discussion on these matters here. 

First of the concerns is the transient induced electrical potential on the screen conductors during the pulsing of the kicker magnet. It is due to this high induced voltage (up to 30kV on the conductor closest to the HV busbar during the field rise and -15kV during the field fall) and the resulting surface flashover between screen conductors and to the end of the beam screen, at the capacitively coupled end, that was the initial motivation to remove the 9 screen conductors from the beam screen. A flashover along the surface of the ceramic, from a screen conductor to ground, can result in the MKI field pulse prematurely falling to a low-level, thus mis-kicking some of the bunches being injected, These mis-kicked bunches can cause significant damage to the downstream equipment.

In addition to improving the screening of the ferrite yoke from the beam any proposed beam screen design must also thus allow the magnet to be pusled without electrical breakdown of the screen. Work by Barnes et al \cite{Barnes:eFieldMKI} has found there are a number of factors that can reduce the electric field on the screen conductors. These include the use of spheres on the ends of the screen conductors to reduce the high electric fields at the ends of the screen conductors (Fig.~\ref{fig:mki8d-points}), tapering the length of the screen conductors (Fig.~\ref{fig:24-alternating-length}) and replacing the outer metallization (at the capacitively coupled end) with a cylindrical metal tube with a 1-3mm vacuum gap between the cylnder and the outer surface of the ceramic tube near the ends of the screen conductors (Fig.~\ref{fig:24-step-out-slight}). The aim of this last measure is to remove the external ground from the outside of the ceramic tube, as the metallization has been observed to effectively force the ground plane closer to the screen conductors due to the high permitivitty of the ceramic tube. The effects of these changes on the beam coupling impedance are discussed in later sections.

The second major concern is the transfer of thermal energy out of the kicker magnets. Due to the HV pulsed nature of the kicker magnet operation there is no active cooling within the device, thus cooling is reliant exclusively on thermal radiation of the heat from the ferrite yoke to the vacuum tank. Experiments have shown that the internal surface of the vacuum tank has a very low emissivity in the IR range \cite{Barnes:emisMKI}, and subsequent thermal simulations have shown that the stable temperature of the MKI is very sensitive to the emissivity of the vacuum tank internal surface \cite{Garlasche:2dHeatEmis}. Subsequently a significant quantity of work has been carried out on methods to improve the emissivity of the internal surface of the vacuum tank so as to more effectively passively cool the device, without degrading the ultrahigh vacuum within the tank.