\subsection{Development of the LHC-MKI Beam Screen}

During the commissioning of the CERN accelerator complex for LHC type beam with 25ns, 50ns and 75ns bunch spacing, drastic heating and outgassing was observed in the SPS extraction kicker magnets (MKEs) [cite PAC paper]. During an extensive measuring campaign, it was deduced that this was due to extensive beam-induced heating due to the large real component of the longitudinal impedance of the MKEs. Subsequently a number of retroactive impedance reduction methods were proposed and evaluated for their effectiveness in reducing the longitudinal impedance[cite Kroyer et co.].

These experiences motivated some debate on the need for a beam screen for the LHC-MKIs [cite Vos paper/Caspers], which ultimately was implemented in the final design. The beam screen is composed of a ceramic pipe housing up to 24 screen conductors - conductive wires which extend the length of the magnet, connected directly to the LHC beam pipe at one end, and capacitively coupled at the other. A scheme involving printed conductive strips on the inside surface of the ceramic was considered but was not implemented due to extensive tracking and electrical breakdown between the strips during firing of the kickers. The capacitive coupling is required due to the short rise time requirement of the kicker field - direct connections at both ends would create a faraday cage, causing the characteristic rise time of the system to dramatically increase. In this case electrical simulations indicate that this must be the pulse input end, i.e. the upstream end of the kicker[cite the beam screen mki]. This will become relevent during the evaluation of heating patterns of the MKI during operation.

The use of the capactive coupling gives rise to a discontinuity of the conducting path of the beam image current. This gives rise to the possibility of exciting resonances in the surrounding structure of the image current path. In this case, resonant modes of a characteristic frequency due to the screen conductors acting as $n \lambda /4$ resonators. To damp these low frequency modes (the screen conductors are some 3m in length), a series of ferrite (NiZn) toroids is placed at each end of the beam screen to damp these modes [cite]. 

During testing and conditioning of the MKIs it was discovered than there was still significant electrical breakdown/sparking occuring during firing. Electrical simulations and measurements indicated that above a PFN voltage of 30kV large quantities of discharge were observed originating from the screen conductors. In particular, the highest voltage was found to occur on the screen conductors closest to the HV busbar [cite improved beam screen].