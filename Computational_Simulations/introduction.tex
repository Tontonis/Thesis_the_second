Whilst beam-based measurements and bench-top measurements techniques have been used for some time to measure the impedance of devices and machines, the use of numerical codes to solve 2D and 3D structures is relatively young as a method of identifying the impedance of devices. Recent progress in computational power has now made these a powerful tool in the regime of impedance evaluation. Codes exist that solve simple 2D structures (ECHO2D \cite{Weiland:Echo2d}, ABCI \cite{Chin:ABCI}), 3D structures (CST Particle Studio \cite{cst}, HFSS \cite{hfss}, Maxwell 3D \cite{maxwell}, MAFIA \cite{Weiland:MAFIAv4}) and 3D structures using highly parallelised codes (GdFidl \cite{gdfidl}, ACE3P \cite{Ng:Ace3p}) which allow the simulation of large, complex structures. The codes may be seperated in to two seperate families; time-domain, which calculate the EM fields in a structure by solving Maxwell's equations in the time domain due to a signal inpulse, and frequency-domain which can be used in a number of ways to simulate the beam coupling impedance.

Each of these families of codes has there own relative advantages and disadvantages and peculiarities to use. In this chapter there will a general introduction to a number of the techniques that may be used to calculate the beam coupling impedance from both time-domain and frequency-domain simulations, along with the limitations of each method.