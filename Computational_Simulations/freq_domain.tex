\section{Frequency Domain Simulations}

There are a number of methods of analysing the beam coupling impedance of a structure using frequency domain simulations. These involve evaluating the structure with no beam present via the use of eigenmode solvers, simulating measurement setups, particularly the coaxial wire technique for systems in which the impedance structure is largely unknown. Finally it is possible to simulate the particle beam also, however this requires a high resolution frequency scan for structures that are expected to be resonant in nature.

\subsection{Eigenmode Simulations}

Eigenmode solvers are a subset of frequency domain solver that are used to identify strongly resonant modes in a structure. These can be cavity modes, antenna like oscillations or a number of other resonances within the structure. The resulting output of the simulation is typically the resonant frequency of the eigenmode(s), the quality factor $Q$ of the mode (if lossy boundary materials are defined) and the field pattern of the eigenmode solution.

From the field pattern it is possible to readily calculate the longitudinal and transverse $R/Q$ of each cavity mode as defined in Sec.~\ref{sec:imp_geo_imp}. The field patterns may also be used to calculate a number of other properties for each eigenmode, such as surface losses and stored energy in the cavity, which will be explained in further details in Sec.~\ref{sec:damping_materials}. Further details on this method are available in \cite{Grudiev:LongTransSecCol}.

\subsection{The Coaxial Wire Method by Simulation}

The coaxial wire method as described in Sec.~\ref{sec:coax_wire_meth} can be directly simulated using waveguide ports to represent matched connections at the ends of the device under test. The resulting simulations provide a transmission coefficient $S_{21}$ which may subsequently be evaluated in the same manner as measurements made with a physical wire. As with the measurements used in practice, a displaced wire and two wires may be simulated and again treated as measurements. 

\subsection{Simulation of the particle beam}

It is also possible to simulate a particle beam directly in the frequency domain. This is done by defining a wave source that produces a field similar to that of the particle beam. For ultrarelativistic beams this neccessitates a source field that is tangential to the direction of motion, and this may technically be possible for cases in which $\beta < 1$. The field components due to the source may be acquired and are the equivalent of the wakefield contribution at a given frequency. The beam impedance $Z$ can subsequently be evaluated from the resulting fields. Further details of this method may be seen in \cite{Kononenko:TransBeamLoading}.

