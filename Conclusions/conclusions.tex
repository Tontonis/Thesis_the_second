\label{chap:Con}

\section{Summary}

In this thesis has been presented a number of studies into the impedance reduction techniques employed in two key systems of LHC, the collimation systems and the injection kicker magnets. Problems with these systems have been examined, focusing on their shortcomings from an operational standpoint especially the high temperatures experienced due to beam-induced heating, using both computational and bench-top measurements to identify and characterise the beam-coupling impedance responsible for this. As part of this a new method for measuring the transverse beam impedances of asymmetric structures has been derived and verified using simulated measurements in comparison to an analytical model of an asymmetric structure (in this case C-core ferrite kicker). Applications of this method to bench top measurements of an injection kicker magnet in the LHC show promise for the method when compared to simulated impedances of the same structure. In addition a study of the power loss in ferrite damped cavities with different extents of damping and screening of the ferrite has been studied in order to better understand the heat loads in structures with a complex structure which may limit the effective damping of cavity modes by ferrite.

Subsequently improvements to the existing systems have been proposed, with a focus on the expected power load due to beam-induced heating, resulting in a new RF damping system for use in the TCTP collimators, and possibly for further use in the phase 2 secondary collimators. An intermediate improvement to the beam screen in the LHC injection kickers was proposed and implemented in MKI8d during technical stop 3 (23/09/12-27/09/12) which contributed to greatly reducing the measured temperature in this magnet. Following from this a further improved beam screen design for use in the LHC injection kicker magnets has been implemented: this will be constructed, detailed measurements carried out to verify the predicted beam coupling impedance, tested for electric field surface flashover and installed during Long Shutdown 1 in 2013/14. This is expected to reduce the power deposition in the magnets by a factor of 4-5 after LS1 when compared to operation 2012, from 150W per magnet for 50ns bunch spacing, to 30W per magnet for 25ns bunch spacing. This improved beam screen has achieved better screening by modifying the layout at the capacitively coupled end to to allow complete coverage of the ferrite yoke, as opposed to only partial coverage as was the case with the old screen design. This necessitated the alteration of the capacitvely coupled end to introduce a step away from the surface of the ceramic tube near the ends of the screen conductors which served to greatly reduce the electric field strength in this region.

\section{Future Directions}

It has become clear during this work that with the increasing high beam currents demanded at high energy hadron colliders, in addition to the very long fill times, that beam-induced heating must seriously be considered as a potential limitation on the integrated luminosity delivered to the experiments due to both high temperatures that may damage equipment, or due to long cool down times necessary between the fills. Beam-induced heating has been observed previously in lepton machines (for example PEP-II at SLAC \cite{Pivi:PEP}) where the beam power spectrum is Gaussian in profile (due to synchrotron radiation) and extends to many tens of gigahertz. Hadron colliders conversely have a power spectrum that extends to a couple of gigahertz, and typically non-Gaussian profiles. This causes higher order lobes to form in the power spectrum which cause complicated interactions with beam impedance at high frequencies which is heavily dependent on the bunch length in the machine. During operation of the LHC in 2011 and 2012 numerous examples of equipment damaged by beam induced heating has been observed, and in this thesis has been presented a study of the LHC injection kicker magnets, in which a new beam screen design was chosen to be implemented in the after LS1 to combat this heating. Many of these devices have subsequently been examined and solutions proposed to reduce the heating. With the increased beam current proposed for both post-LS1 and HL-LHC in the future, careful study of the beam coupling impedance and the resulting instabilities and heating will remain an important issue for the coming future.