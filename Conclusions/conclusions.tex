In this thesis has been presented a number of studies into the impedance reduction techniques employed in two key systems in LHC, the injection kicker magnets and the collimation systems. Problems with these existing methods have been examined, focusing on their shortcomings from an operational standpoint especially the high temperatures experienced due to beam-induced heating, using both computational and bench-top measurements to identify and charcterise the beam-coupling impedance responsible for this. As part of this a new method for measuring the transverse beam impedances of asymmetric structures has been proposed and verified using experimental and computational measurements against both an analytical model and a measured structure. In addition a study of the power loss in ferrite damped cavities with different extents of damping and screening of the ferrite has been presented in order to better understand the heat loads in structures with a complex structure which may limit the effective damping of cavity modes by ferrite.

Subsequently improvements to the existing systems have been proposed, with a certain focus on the expected power load due to beam-induced heating, resulting in a new RF damping system for use in the TCTP collimators, and possibly for further use in the phase 2 secondary collimators and an improved beam screen for use in the LHC injection kicker magnets to be constructed and installed during LS1 in 2013/14.

It has become clear during this work that with the increasing high beamcurrents demanded at high energy hadron colliders in addition to the very long fill times that beam-induced heating must seriously be considered as a potential limitation on the integrated luminosity delivered to the experiments due to both high temperatures that may damage equipment, or due to long cool down times necessary between uses. As such knowledge of the existing impedance reduction techniques must be propogated to the accelerator community, and the suitability and correct implementation of them studied thoroghly during the \emph{design} phase of equipment design can not be emphasised enough.